\documentclass{beamer}
\usetheme{Gelugor}
\usepackage{hyperref}
\usepackage[utf8]{inputenc}
\usepackage[spanish]{babel}
\usepackage[default]{droidserif}
\title{Análisis y Diseño de Software}
\subtitle{Aquí el tema sobre tu trabajo}
\author{Tu Nombre}
\date{\today}
\institute{Universidad Nacional de Loja \\ tucorreo@unl.edu.ec}

\begin{document}
	
	%INICIO carátula
	\begin{frame}[plain,t]
		\titlepage
	\end{frame}
	%FIN carátula

	\section{Tema}

	\subsection{Introducción}
		\begin{frame}
			\frametitle{Introducción}
			%\framesubtitle{Tema}
			Lorem ipsum ad his scripta blandit partiendo, eum fastidii accumsan euripidis in, eum liber hendrerit an. Qui ut wisi 					vocibus suscipiantur, quo dicit ridens inciderint id. Quo mundi lobortis reformidans eu, legimus senserit 								definiebas an eos. Eu sit tincidunt incorrupte definitionem, vis mutat affert percipit cu, eirmod 										consectetuer signiferumque eu per. In usu latine equidem 		dolores. Quo no falli viris intellegam, ut fugit veritus 					placerat per.
\end{frame}

\subsection{Agenda}
\begin{frame}
\frametitle{Agenda}
\begin{itemize}
\item Item 1
	\begin{itemize}
	\item sub item
	\item sub item
	\end{itemize}
\item Item 2
	\begin{enumerate}
	\item Punto 1
	\end{enumerate}
\item Item 3
	\begin{description}
	\item[Nota] xyz
	\end{description}
\end{itemize}
\end{frame}

\subsection{Objetivos}
\begin{frame}
\frametitle{Objetivos}
%\begin{definition}[Greetings]
	\begin{itemize}
		\item 1 x
		\item 2 y
		\item 3 z
	\end{itemize}
%\end{definition}

%\begin{theorem}[Fermat's Last Theorem]
%$a^n + b^n = c^n, n \leq 2$
%\end{theorem}

%\begin{alertblock}{Uh-oh.}
%By the pricking of my thumbs.
%\end{alertblock}

%\begin{exampleblock}{Uh-oh.}
%Something evil this way comes.
%\end{exampleblock}

\end{frame}


\subsection{Desarrollo}
\begin{frame}
\frametitle{Desarrollo}
Lorem ipsum ad his scripta blandit partiendo, eum fastidii accumsan euripidis in, eum liber hendrerit an. Qui ut wisi vocibus suscipiantur, quo dicit ridens inciderint id. Quo mundi lobortis reformidans eu, legimus senserit definiebas an eos. Eu sit tincidunt incorrupte definitionem, vis mutat affert percipit cu, eirmod consectetuer signiferumque eu per. In usu latine equidem dolores. Quo no falli viris intellegam, ut fugit veritus placerat per.
\end{frame}


\subsection{Conclusiones}
\begin{frame}
\frametitle{Conclusiones}
Lorem ipsum ad his scripta blandit partiendo, eum fastidii accumsan euripidis in, eum liber hendrerit an. Qui ut wisi vocibus suscipiantur, quo dicit ridens inciderint id. Quo mundi lobortis reformidans eu, legimus senserit definiebas an eos. Eu sit tincidunt incorrupte definitionem, vis mutat affert percipit cu, eirmod consectetuer signiferumque eu per. In usu latine equidem dolores. Quo no falli viris intellegam, ut fugit veritus placerat per.
\end{frame}

\subsection{Bibliografia}
\begin{frame}
\frametitle{Bibliografia}
	\begin{itemize}
		\item INGENIERÍA DEL SOFTWARE UN ENFOQUE PRÁCTICO Quinta edición
		\item INGENIERÍA DEL SOFTWARE UN ENFOQUE PRÁCTICO Quinta edición
		\item INGENIERÍA DEL SOFTWARE UN ENFOQUE PRÁCTICO Quinta edición
	\end{itemize}
\end{frame}

\subsection{Licencia}
\begin{frame}
\frametitle{Licencia}
\begin{center}
\href{http://www.google.com}{\includegraphics[scale=.8]{cc}}
\end{center}
\end{frame}

\MuchasGraciasFrame

\end{document}